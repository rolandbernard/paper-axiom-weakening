%% The first command in your LaTeX source must be the \documentclass command.
%%
%% Options:
%% twocolumn : Two column layout.
%% hf: enable header and footer.
\documentclass[
% twocolumn,
% hf,
]{ceurart}

%%
%% One can fix some overfulls
\sloppy

%%
%% Minted listings support 
%% Need pygment <http://pygments.org/> <http://pypi.python.org/pypi/Pygments>
\usepackage{listings}
%% auto break lines
\lstset{breaklines=true}

%%
%% DL Logo for inline use
\usepackage{graphbox}
\DeclareRobustCommand{\DLLogo}{%
  \begingroup\normalfont
  \kern-1.75pt\includegraphics[align=c,height=1.25\baselineskip]{dl}\kern-1.5pt%
  \endgroup
}

%%
%% AMS Theorems
\usepackage{amsthm}
\newtheorem{theorem}{Theorem}
\newtheorem{definition}{Definition}
\newtheorem{example}{Example}

%%%%%%%%%%%%%%%%%%%%%%%%%%%%%%%%%%%%%%%%
%%% OUR SETTINGS
\usepackage{amsmath}
\usepackage{amssymb}
\usepackage{mathtools}
\usepackage{algorithm}
\usepackage{algpseudocode}
% \usepackage{theorem}
\usepackage{xspace}
\usepackage{dl}
\usepackage{url}
%\usepackage[numbers]{natbib}
\clubpenalty = 10000
\widowpenalty = 10000
\displaywidowpenalty = 10000
% theorems and the like
% \newtheorem{theorem}{Theorem}
\newtheorem{lemma}{Lemma}
% \newtheorem{proposition}{Proposition}
% \newtheorem{corollary}{Corollary}

% \theorembodyfont{\rmfamily}
% \newtheorem{definition}{Definition}
% \newtheorem{example}{Example}

% \theorembodyfont{\slshape}
% \newtheorem{assumption}{Assumption}

% \theorembodyfont{\slshape}
% \newtheorem{hypothesis}{Hypothesis}

% \newenvironment{justification}{\textsf{Justification:}}{\hfill $\Box$}
%\newenvironment{proof}{\noindent\textsc{proof:}}{\hfill $\square$ \bigskip}


\newcommand{\AL}{\ensuremath{\mathcal{AL}}\xspace}
\newcommand{\ALC}{\ensuremath{\mathcal{ALC}}\xspace}
\newcommand{\SROIQ}{\ensuremath{\mathcal{SROIQ}}\xspace}
\newcommand{\onto}[1]{\ensuremath{\mathsf{#1}}}
\usepackage[colorinlistoftodos]{todonotes}

\newcommand{\cg}
{\ensuremath{\blacktriangle}\xspace}
%{\ensuremath{\dot{\sqcup}}\xspace}

\newcommand{\todoT}[1]{\todo[fancyline,size=\small,color=orange!40]{\textbf{TBD:} #1}\xspace}
\newcommand{\todoR}[1]{\todo[fancyline,size=\small,color=orange!40]{\textbf{rc:} #1}\xspace}

\newcommand{\todoO}[1]{\todo[fancyline,size=\small,color=red!20]{\textbf{ok:} #1}\xspace}

\newcommand{\tododo}[1]{\todo[fancyline,size=\small,color=red!60]{\textbf{ToDO:} #1}\xspace}
\newcommand{\todoin}[1]{\todo[inline,size=\small,color=green!20]{\textbf{HERE:} #1}\xspace}


%\usepackage{pgf}https://preview.overleaf.com/public/pqghhhhwjqpw/images/986e3685291cb332a5fbe5a979b5045d2e649a43.jpeg
\usepackage{tikz}
%\usetikzlibrary{arrows,automata}
\usetikzlibrary{positioning,calc,backgrounds}

%%%%%%%MACROS%%%%%%%%
\newcommand{\Lmc}{\ensuremath{\mathcal{L}}\xspace}
\newcommand{\Imc}{\ensuremath{\mathcal{I}}\xspace}
\newcommand{\Jmc}{\ensuremath{\mathcal{J}}\xspace}
\newcommand{\Tmc}{\ensuremath{\mathcal{T}}\xspace}
\newcommand{\Amc}{\ensuremath{\mathcal{A}}\xspace}
\newcommand{\Rmc}{\ensuremath{\mathcal{R}}\xspace}
\newcommand{\Omc}{\ensuremath{\mathcal{O}}\xspace}
\newcommand{\Omcref}{\ensuremath{{\mathcal{O}^\textnormal{ref}}}\xspace}
\newcommand{\Omcfull}{\ensuremath{{\mathcal{O}^\textnormal{full}}}\xspace}
\newcommand{\Ontology}{\Omc}

\newcommand{\EL}{\ensuremath{\mathcal{E\!L}}\xspace}
\newcommand{\elpp}{\ensuremath{\mathcal{E\!L}^{++}}\xspace}

\newcommand{\Inf}{{\ensuremath{\mathsf{Inf}}}\xspace}
\newcommand{\qual}{{\ensuremath{\mathsf{IIC}}}\xspace}

\newcommand{\UpC}{{\ensuremath{\mathsf{UpCover}}}\xspace}
\newcommand{\DownC}{{\ensuremath{\mathsf{DownCover}}}\xspace}

\newcommand{\disjoint}{\ensuremath{\mathit{disjoint}}\xspace}
\newcommand{\self}{\ensuremath{\mathit{Self}}\xspace}
\newcommand{\less}[2]{\ensuremath{\leq #1~#2}\xspace}
\newcommand{\more}[2]{\ensuremath{\geq #1~#2}\xspace}
\newcommand{\nominal}[1]{\ensuremath{\{#1\}}\xspace}

\newcommand{\inv}{\ensuremath{\mathit{inv}}\xspace}
\newcommand{\refine}{\ensuremath{\mathop{\uparrow}}\xspace}
\newcommand{\corefine}{\ensuremath{\mathop{\downarrow}}\xspace}
%\DeclareMathOperator*{\argmax}{\mathsf{arg\,max}}
\DeclareMathOperator*{\argmax}{\mathsf{argmax}}

%%%%%%%%%%%%%%%%%%%%%%%%%%%%%%%%%%%%%%%%

%%
%% end of the preamble, start of the body of the document source.
\begin{document}

%%
%% Rights management information.
%% CC-BY is default license.
\copyrightyear{2023}
\copyrightclause{Copyright for this paper by its authors.
  Use permitted under Creative Commons License Attribution 4.0
  International (CC BY 4.0).}

%%
%% This command is for the conference information
\conference{\DLLogo{} DL 2023: 36th International Workshop on Description Logics,
  September 2--4, 2023, Rhodes, Greece}

%%
%% The "title" command
\title{Implementing Axiom Weakening for SROIQ}

%%
%% The "author" command and its associated commands are used to define
%% the authors and their affiliations.

\author[1]{Roland Bernard}[
email=roland.bernard@student.unibz.it,
]
\author[1]{Oliver Kutz}[
email=oliver.kutz@unibz.it,
]
\author[1]{Nicolas Troquard}[
email=nicolas.troquard@unibz.it,
]
\address[1]{
Free University of Bozen-Bolzano, Italy
}

%%
%% The abstract is a short summary of the work to be presented in the
%% article.
\begin{abstract}
Axiom weakening is a technique that allows for a fine-grained repair of inconsistent ontologies. Its main advantage is that it repairs ontologies by making axioms
less restrictive rather than by deleting them, employing refinement operators. In this paper, we build on previously introduced axiom weakening for \ALC, and show how it can be extended to deal with \SROIQ, the expressive and decidable description logic underlying OWL 2 DL.
We here focus on describing a prototype implementation computing axiom weakening for \SROIQ and discuss a number of performance and evaluation aspects.
%the definitions of the refinement operators to deal with \SROIQ constructs, in particular with %such as reflexive and irreflexive roles, disjoint roles, 
%role hierarchies,  cardinality constraints and nominals, and illustrate its application. Finally, we discuss the problem of termination of an iterated weakening procedure.
\end{abstract}

%%
%% Keywords. The author(s) should pick words that accurately describe
%% the work being presented. Separate the keywords with commas.
\begin{keywords}
  Description Logic \sep
  Knowledge refinement \sep
  Prot\'eg\'e
\end{keywords}

%%
%% This command processes the author and affiliation and title
%% information and builds the first part of the formatted document.
\maketitle

\section{Introduction: Weakening for debugging}

% motivate repair using axiom weakening
% give an example ontology where classical repairs are insufficient
% give an example where it is advantageous to weaken the role hierarchy
% explain other gentle repair approaches
% briefly describe refinement operators
% briefly describe what we have done

\section{Preliminaries}

\section{Axiom Weakening for \ALC}

% describe the ALC syntax and semantics
% define sub concepts
% define upward and downward covers
% give an example of some cover applications
% define the abstract and concrete refinement operators
% give an example of the concrete refinement operators
% define the weakening operator
% give an example of the weakening operator

\section{Extending Weakening to \SROIQ}

% describe the SROIQ syntax and semantics
% describe regularity since it is relevant
% explain the complications of weakening SROIQ

\begin{example}
    Some examples for some \SROIQ features.
\end{example}

% lay out the constraints that we want to maintain
% explain how each of them can be ensured
% define upward and downward covers
% define upward and downward covers
% proof that the constrains are retained

\section{Implementing Axiom Weakening for \SROIQ}

% explain how the weakening is used for repair
% explain how the reference ontology is selected
% explain how bad axioms are selected

\section{Weakening makes you strong: evaluation aspects}

% explain how different repairs are compared
% explain how inconsistent ontologies were generated
% explain that the OWL 2 ontologies were normalized to only use SROIQ axioms
% show results of the evaluation

\section{Outlook}

% complex roles in up and down covers
% more permissive refinement of role inclusions
% working directly with OWL 2 axioms
% how to steer the repair process

%%
%% Define the bibliography file to be used
\bibliography{biblio}

%%
%% If your work has an appendix, this is the place to put it.
\appendix


\end{document}

%%
%% End of file
