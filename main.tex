%% The first command in your LaTeX source must be the \documentclass command.
%%
%% Options:
%% twocolumn : Two column layout.
%% hf: enable header and footer.
\documentclass[
% twocolumn,
% hf,
]{ceurart}

%%
%% One can fix some overfulls
\sloppy

%%
%% Minted listings support 
%% Need pygment <http://pygments.org/> <http://pypi.python.org/pypi/Pygments>
\usepackage{listings}
%% auto break lines
\lstset{breaklines=true}

%%
%% DL Logo for inline use
\usepackage{graphbox}
\DeclareRobustCommand{\DLLogo}{%
  \begingroup\normalfont
  \kern-1.75pt\includegraphics[align=c,height=1.25\baselineskip]{dl}\kern-1.5pt%
  \endgroup
}

%%
%% AMS Theorems
\usepackage{amsthm}
\newtheorem{theorem}{Theorem}
\newtheorem{definition}{Definition}
\newtheorem{example}{Example}

%%%%%%%%%%%%%%%%%%%%%%%%%%%%%%%%%%%%%%%%
%%% OUR SETTINGS
\usepackage{cleveref}
\usepackage{amsmath}
\usepackage{amssymb}
\usepackage{mathtools}
\usepackage{algorithm}
\usepackage{algpseudocode}
% \usepackage{theorem}
\usepackage{xspace}
\usepackage{dl}
\usepackage{url}
%\usepackage[numbers]{natbib}
\clubpenalty = 10000
\widowpenalty = 10000
\displaywidowpenalty = 10000
% theorems and the like
% \newtheorem{theorem}{Theorem}
\newtheorem{lemma}{Lemma}
% \newtheorem{proposition}{Proposition}
% \newtheorem{corollary}{Corollary}

% \theorembodyfont{\rmfamily}
% \newtheorem{definition}{Definition}
% \newtheorem{example}{Example}

% \theorembodyfont{\slshape}
% \newtheorem{assumption}{Assumption}

% \theorembodyfont{\slshape}
% \newtheorem{hypothesis}{Hypothesis}

% \newenvironment{justification}{\textsf{Justification:}}{\hfill $\Box$}
%\newenvironment{proof}{\noindent\textsc{proof:}}{\hfill $\square$ \bigskip}


\newcommand{\AL}{\ensuremath{\mathcal{AL}}\xspace}
\newcommand{\ALC}{\ensuremath{\mathcal{ALC}}\xspace}
\newcommand{\SROIQ}{\ensuremath{\mathcal{SROIQ}}\xspace}
\newcommand{\onto}[1]{\ensuremath{\mathsf{#1}}}
\usepackage[colorinlistoftodos]{todonotes}

\newcommand{\cg}
{\ensuremath{\blacktriangle}\xspace}
%{\ensuremath{\dot{\sqcup}}\xspace}

\newcommand{\todoT}[1]{\todo[fancyline,size=\small,color=orange!40]{\textbf{TBD:} #1}\xspace}
\newcommand{\todoR}[1]{\todo[fancyline,size=\small,color=orange!40]{\textbf{rc:} #1}\xspace}

\newcommand{\todoO}[1]{\todo[fancyline,size=\small,color=red!20]{\textbf{ok:} #1}\xspace}

\newcommand{\tododo}[1]{\todo[fancyline,size=\small,color=red!60]{\textbf{ToDO:} #1}\xspace}
\newcommand{\todoin}[1]{\todo[inline,size=\small,color=green!20]{\textbf{HERE:} #1}\xspace}


%\usepackage{pgf}https://preview.overleaf.com/public/pqghhhhwjqpw/images/986e3685291cb332a5fbe5a979b5045d2e649a43.jpeg
\usepackage{tikz}
%\usetikzlibrary{arrows,automata}
\usetikzlibrary{positioning,calc,backgrounds}

%%%%%%%MACROS%%%%%%%%
\newcommand{\Lmc}{\ensuremath{\mathcal{L}}\xspace}
\newcommand{\Imc}{\ensuremath{\mathcal{I}}\xspace}
\newcommand{\Jmc}{\ensuremath{\mathcal{J}}\xspace}
\newcommand{\Tmc}{\ensuremath{\mathcal{T}}\xspace}
\newcommand{\Amc}{\ensuremath{\mathcal{A}}\xspace}
\newcommand{\Rmc}{\ensuremath{\mathcal{R}}\xspace}
\newcommand{\Omc}{\ensuremath{\mathcal{O}}\xspace}
\newcommand{\Omcref}{\ensuremath{{\mathcal{O}^\textnormal{ref}}}\xspace}
\newcommand{\Omcfull}{\ensuremath{{\mathcal{O}^\textnormal{full}}}\xspace}
\newcommand{\Ontology}{\Omc}

\newcommand{\EL}{\ensuremath{\mathcal{E\!L}}\xspace}
\newcommand{\elpp}{\ensuremath{\mathcal{E\!L}^{++}}\xspace}

\newcommand{\Inf}{{\ensuremath{\mathsf{Inf}}}\xspace}
\newcommand{\qual}{{\ensuremath{\mathsf{IIC}}}\xspace}

\newcommand{\UpC}{{\ensuremath{\mathsf{UpCover}}}\xspace}
\newcommand{\DownC}{{\ensuremath{\mathsf{DownCover}}}\xspace}

\newcommand{\disjoint}{\ensuremath{\mathit{disjoint}}\xspace}
\newcommand{\self}{\ensuremath{\mathit{Self}}\xspace}
\newcommand{\less}[2]{\ensuremath{\leq #1~#2}\xspace}
\newcommand{\more}[2]{\ensuremath{\geq #1~#2}\xspace}
\newcommand{\nominal}[1]{\ensuremath{\{#1\}}\xspace}

\newcommand{\inv}{\ensuremath{\mathit{inv}}\xspace}
\newcommand{\refine}{\ensuremath{\mathop{\uparrow}}\xspace}
\newcommand{\corefine}{\ensuremath{\mathop{\downarrow}}\xspace}
%\DeclareMathOperator*{\argmax}{\mathsf{arg\,max}}
\DeclareMathOperator*{\argmax}{\mathsf{argmax}}

%%%%%%%%%%%%%%%%%%%%%%%%%%%%%%%%%%%%%%%%

%%
%% end of the preamble, start of the body of the document source.
\begin{document}

%%
%% Rights management information.
%% CC-BY is default license.
\copyrightyear{2023}
\copyrightclause{Copyright for this paper by its authors.
  Use permitted under Creative Commons License Attribution 4.0
  International (CC BY 4.0).}

%%
%% This command is for the conference information
\conference{\DLLogo{} DL 2023: 36th International Workshop on Description Logics,
  September 2--4, 2023, Rhodes, Greece}

%%
%% The "title" command
\title{Implementing Axiom Weakening for SROIQ}

%%
%% The "author" command and its associated commands are used to define
%% the authors and their affiliations.

\author[1]{Roland Bernard}[
email=roland.bernard@student.unibz.it,
]
\author[1]{Oliver Kutz}[
email=oliver.kutz@unibz.it,
]
\author[1]{Nicolas Troquard}[
email=nicolas.troquard@unibz.it,
]
\address[1]{
Free University of Bozen-Bolzano, Italy
}

%%
%% The abstract is a short summary of the work to be presented in the
%% article.
\begin{abstract}
Axiom weakening is a technique that allows for a fine-grained repair of inconsistent ontologies. Its main advantage is that it repairs ontologies by making axioms less restrictive rather than by deleting them, employing refinement operators. In this paper, we build on previously introduced axiom weakening for \ALC, and show how it can be extended to deal with \SROIQ, the expressive and decidable description logic underlying OWL 2 DL. We here focus on describing a prototype implementation computing axiom weakening for \SROIQ and discuss a number of performance and evaluation aspects.
%the definitions of the refinement operators to deal with \SROIQ constructs, in particular with %such as reflexive and irreflexive roles, disjoint roles, role hierarchies,  cardinality constraints and nominals, and illustrate its application. Finally, we discuss the problem of termination of an iterated weakening procedure.
\end{abstract}

%%
%% Keywords. The author(s) should pick words that accurately describe
%% the work being presented. Separate the keywords with commas.
\begin{keywords}
  Description Logic \sep
  Knowledge refinement \sep
  Prot\'eg\'e
\end{keywords}

%%
%% This command processes the author and affiliation and title
%% information and builds the first part of the formatted document.
\maketitle

\section{Introduction: Weakening for debugging}

% \begin{example}
% % give an example ontology where classical repairs are insufficient
% % give an example where it is advantageous to weaken the role hierarchy
% \end{example}

% motivate repair using axiom weakening
% explain other gentle repair approaches
% briefly describe refinement operators
% briefly describe what we have done

% \section{Axiom Weakening for \ALC}

% % describe the ALC syntax and semantics
% % define sub concepts
% % define upward and downward covers

% \begin{example}
% % give an example of some cover applications
% \end{example}

% % define the abstract and concrete refinement operators

% \begin{example}
% % give an example of the concrete refinement operators
% \end{example}

% % define the weakening operator

% \begin{example}
% % give an example of the weakening operator
% \end{example}

\section{Extending Weakening to \SROIQ}

% describe the SROIQ syntax
% describe roles and concepts
We now give a brief description of the DL \SROIQ; for full details, see \cite{baader_horrocks_lutz_sattler_2017,HorrocksKutzSattlerKR2006}. The syntax of \SROIQ is based on a vocabulary of three disjoint sets $N_C$, $N_R$, $N_I$ of respectively \emph{concept names}, \emph{role names}, and \emph{individual names}. The set of \SROIQ \emph{concepts} and \emph{roles} is generated by the following grammar.
\begin{alignat*}{2}
  R, S &::={} && U \mid E \mid r \mid r^{-} \enspace,\\
  C &::= && \bot \mid \top \mid A \mid \neg C \mid C \sqcap C \mid C \sqcup C \mid \forall R.C \mid \exists R.C \mid \\ 
  &&& \more n S.C \mid \less n S.C \mid \exists S.\self \mid \nominal i \enspace,
\end{alignat*}
where $A \in N_C$ is a concept name, $r \in N_R$ is a role name, $i \in N_I$ is an individual name and $n \in \mathbb{N}_0$ is a non-negative integer. $U$ and $E$ are respectively the universal role and existential role. $S$ is a \emph{simple role} (see below) in the RBox $\Rmc$. In the following, $\Lmc(N_C, N_R, N_I)$ and $\Lmc(N_R) = N_R \cup \{U, E\} \cup \{r^- \mid r \in N_R\}$ denote respectively the set of concepts and roles that can be built over $N_C$, $N_R$, and $N_I$ in \SROIQ.

% describe the TBox, ABox and RBox statements
A \emph{TBox} $\Tmc$ is a finite set of concept inclusions (GCIs) of the form $C \sqsubseteq D$ where $C$ and $D$ are concepts. The TBox is used to stores terminological knowledge concerning the relationship between concepts. A \emph{ABox} $\Amc$ is a finite set of statements of the form $R(a)$, $\lnot R (a)$, $a = b$, and $a \not= b$ where $R$ is a role and $a$ and $b$ are individual names. The ABox expresses knowledge regarding individuals in the domain. A \emph{RBox} $\Rmc$ is a finite set of role inclusions (RIAs) of the form $R_1 \circ \cdots \circ R_n \sqsubseteq R$, and disjoint role axioms $\disjoint(S_1, S_2)$ where $R$, $R_1$, $\dots$, $R_n$, $S_1$, and $S_2$ are roles. $S_1$ and $S_2$ are simple (see next) in the RBox $\Rmc$. The special case of $n = 1$ is a simple role inclusion, while we call the cases where $n > 1$ complex role inclusions. The RBox represents knowledge about the relationship between roles.

% describe simple and complex roles since it is relevant
The set of \emph{non-simple} roles in $\Rmc$ is the smallest set such that: $U$ and $E$ are non-simple; any role $R$ that appears as the super role of a complex RIA $R_1 \circ \cdots \circ R_n \sqsubseteq R$ where $n > 1$ is non-simple; any role $R$ that appears on the right-hand side of a simple RIA $S \sqsubseteq R$ where $S$ is non-simple, is also non-simple; and a role $r$ is non-simple if and only if $r^-$ is non-simple.
All other roles are \emph{simple}.

% describe regularity since it is relevant
For convenience, let us define the function $\inv(R)$ such that $\inv(r) = r^-$ and $\inv(r^-) = r$ for all role names $r \in N_R$. A RBox $\Rmc$ is \emph{regular} if there exists a pre-order $\preceq$, i.e., a transitive and reflexive relation, over the set of roles such that $R \preceq S \iff \inv(R) \preceq \inv(S)$, $R \preceq S \iff \inv(R) \preceq S$, and all RIAs in $\Rmc$ are of the forms:
$\inv(R) \sqsubseteq R$,
$R \circ R \sqsubseteq R$,
$S \sqsubseteq R$, $R \circ S_1 \circ \cdots \circ S_n \sqsubseteq R$,
$S_1 \circ \cdots \circ S_n \circ R \sqsubseteq R$, or
$S_1 \circ \cdots \circ S_n \sqsubseteq R$,
where $r \in N_R$ is a role name, $n > 1$ and $R$, $S$, $S_1, \cdots, S_n$ are roles such that $S \preceq R$, $S_i \preceq R$, and $R \not\preceq S_i$ for $i = 1, \dots, n$.

A \SROIQ ontology $\Omc = \Tmc \cup \Amc \cup \Rmc$ consists of a TBox $\Tmc$, an ABox $\Amc$, and a RBox $\Rmc$, where $\Rmc$ is regular.

% describe the semantics
% whats a interpretation, when is it a model
The semantics of \SROIQ are defined using \emph{interpretations} $I = \langle \Delta^I, \cdot^I \rangle$ where $\Delta^I$ is a non-empty \emph{domain} and $\cdot^I$ is a function associating to each individual name $a$ an element of the domain $a^I \in \Delta^I$, to each concept $C$ a subset of the domain $C^I \subseteq \Delta^I$, and to each role $R$ a binary relation on the domain $R^I \subseteq \Delta^I \bigtimes \Delta^I$; see \cite{baader_horrocks_lutz_sattler_2017,HorrocksKutzSattlerKR2006} for further details. An interpretation $I$ is a \emph{model} for $\Omc$ if it satisfies all the axioms in $\Omc$.

% describe subsumptions of concepts and roles
Given two concepts $C$ and $D$ we say that $C$ is \emph{subsumed} by $D$ (or $D$ \emph{subsumes} $C$) with respect to the ontology $\Omc$, written $C \sqsubseteq_\Omc D$, if $C^I \subseteq D^I$ in every model $I$ of $\Omc$. Further, $C$ is \emph{strictly subsumed by} $D$, written $C \sqsubset_\Omc D$, if $C \sqsubseteq_\Omc D$ but not $D \sqsubseteq_\Omc C$. Analogously, given two roles $R$ and $S$, $R$ is subsumed by $S$ with respect to $\Omc$ ($R \sqsubseteq_\Omc S$) if $R^I \sqsubseteq S^I$ in all models $I$ of $\Omc$. Again, $R \sqsubset_\Omc S$ holds if $R \sqsubseteq_\Omc S$ but not $D \sqsubseteq_\Omc C$

% explain the complications of weakening SROIQ
The main difficulties that arise when weakening axioms in \SROIQ ontologies, and especially when weakening RIAs, are related to ensuring that the constraints on the use of non-simple roles and the regularity of the role hierarchy are maintained. Not every weaker axiom can be inserted into a valid \SROIQ ontology without causing a violation of these restrictions.

% example of the complications of weakening SROIQ
\begin{example}
  Take the ontology $\Omc = \{ r \circ s \circ r \sqsubseteq t, r \sqsubseteq s, \top \sqsubseteq \forall t.\bot, \exists s.\self \sqsubseteq \top \}$. Since $t$ is empty in every model of this ontology, the axiom $r \sqsubseteq s$ could be weakened to $t \sqsubseteq s$ if we ignore the additional constraints. This would result in an ontology where $s$ is non-simple, which is not allowed since $s$ is used as part of a self constraint.
  Additionally, using this weakening would also cause a non-regular RBox, because for any pre-order $\preceq$, $t \not\preceq s$ must hold for the complex RIA and $t \preceq s$ must hold for the new axiom. Yet, this is a contradiction.
\end{example}

% lay out the constraints that we want to maintain
To prevent these kinds of issues, we restrict how concepts are refined and RIAs weakened. In \cite{confalonieri2020towards} the refinement of RIAs was not considered at all to avoid these problems. In this paper, however, we have extended the axiom weakening operator to handle also RIAs. To achieve this, we must ensure that only simple roles are used when weakening disjoint role axioms or refining cardinality and self constraints. Further, it must be guaranteed that all roles that are currently used in such context remain simple when adding the weakened axioms to the ontology. Finally, the addition of a weakened axiom must maintain the regularity of the role hierarchy. We discuss now the restrictions we applied in order to satisfy these requirements.

% explain how each of them can be ensured
Firstly, the covers and refinement operators for roles operate only on roles that are simple. A similar restriction has already been applied in the refinement operator suggested in \cite{confalonieri2020towards}. Restricting the refinement to simple roles guarantees that the new axioms created by weakening will not contain non-simple roles in axioms or concepts where they are not allowed. An important detail that was not considered in \cite{confalonieri2020towards} is that the roles over which the covers operate must be simple in all ontologies that the weaker axioms are used in. It is therefore not generally sufficient to use the roles that are simple in the reference ontology, since the reference ontology may not contain all RBox axioms, and therefore contain simple roles that are not simple in the full ontology. For this reason, we give to the upward and downward cover as an argument not only the reference ontology $\Omcref$, but also the full ontology $\Omcfull$. Both $\Omcref$ and $\Omcfull$ share the same vocabulary $N_I$, $N_C$, and $N_R$. We assume that $\Omcref \subseteq \Omcfull$. In the context of repairing inconsistent ontologies, $\Omcfull$ can be chosen to be the inconsistent ontology that we want to repair.

Then, to ensure further that by adding weakened axioms we do not cause a constraint violation in existing axioms and concepts, we choose the allowed weakening for RIAs such that all roles that are simple in $\Omcfull$, are also simple after adding to it a weakening of one of its axioms. We observe that for complex RIAs $S_1 \circ \cdots \circ S_n \sqsubseteq R$ we should not refine the role $R$. Since all roles returned by our refinement operator are simple in $\Omcfull$, such a replacement would make a role with was simple in $\Omcfull$ non-simple. A similar argument can be made for refining $R$ in a simple RIA $S \sqsubseteq R$ where the role $S$ is non-simple in $\Omcfull$. So the only way to refine the super role during the weakening of a RIA is when it is a simple RIA and additionally the sub role of the axiom is simple in $\Omcfull$.

When it comes to refining the left-hand side of RIAs, we do not need any special restrictions. The main significant observation is that all roles that are returned by the refinement will be simple. This means that in a simple RIA $R \sqsubseteq S$, even if $S$ is simple, replacing $R$ with another simple role will not cause $S$ to become non-simple. For a complex RIA $S_1 \circ \cdots \circ S_n \sqsubseteq R$ on the other hand, the role $R$ must already have been non-simple in $\Omcfull$, and replacing any $S_i$ with a refinement has no effect on which roles are simple.

A more interesting question is whether such a weakening may still cause a non-regular role hierarchy. The important insight is that simple roles are always allowed on the left-hand side of a RIA. While this is more directly evident in some alternative definitions of regularity (e.g., \cite{rudolph2011foundations}) it is not so apparent from the one presented in this paper. Intuitively, the constraint given above for regularity disallows dependency cycles that contain complex RIAs. Simple roles can not be part of such a cycle, since the cycle must contain at least one complex RIA to be a violation of the constraint, and all roles that depend in this sense on a complex RIA must be non-simple. A more formal justification for this fact is given in the \hyperref[proof:regularity]{proof} for \cref{lem:regularity}.
Since all refinements of the left-hand side of RIAs are performed using simple roles, these can not lead to a non-regular RBox. Further, refinements of the super role of RIAs are only performed on simple RIAs $S \sqsubseteq R$ where $S$ is a simple role. Since $S$ is simple in this case, all refinements of $R$ are allowed, potentially also if the refinement yielded a non-simple role.

\begin{definition}
  Let $\Omc$ be a \SROIQ ontology. The set of \emph{subconcepts} of $\Omc$ is given by 
  \begin{align*}
    \sub(\Omc) = \{\top, \bot\} \cup \bigcup_{C(a) \in \Omc} \sub(C) \cup \bigcup_{C \sqsubseteq D \in \Omc} ( \sub(C) \cup \sub(D) ) \enspace,
  \end{align*}
  where $\sub(C)$ is the set of \emph{subconcepts} in $C$ such that
  \begin{align*}
    \sub(A) &= \{ A \} \quad , A \in N_C \cup \{ \top, \bot \} \enspace, &
    \sub(\lnot C) &= \{ \lnot C \} \cup \sub(C) \enspace, \\
    \sub(C \sqcup D) &= \{ C \sqcup D \} \cup \sub(C) \cup \sub(D) \enspace, &
    \sub(\forall R.C) &= \{ \forall R.C \} \cup \sub(C) \enspace, \\
    \sub(C \sqcap D) &= \{ C \sqcap D \} \cup \sub(C) \cup \sub(D) \enspace, &
    \sub(\exists R.C) &= \{ \exists R.C \} \cup \sub(C) \enspace, \\
    \sub(\more n R.C) &= \{ \more n R.C \} \cup \sub(C) \enspace, &
    \sub(\less n R.C) &= \{ \less n R.C \} \cup \sub(C) \enspace, \\
    \sub(\exists R.\self) &= \{ \exists R.\self \} \enspace, &
    \sub(\nominal{i}) &= \{ \nominal{i} \} \enspace.
  \end{align*}
\end{definition}

% define upward and downward covers
We will define now the upward and downward cover sets for concepts and roles. Intuitively, for a given concept the upward cover is the set of the most specific generalizations from the set of subconcepts or roles, while the downward cover set contains the most general specializations from the same set of subconcepts and roles. We define the upward and downward cover additionally also for non-negative integers, as they will be useful in the refinement of cardinality constraints.

\begin{definition} \label{def:covers}
  Let $\Omcfull$ and $\Omcref \subseteq \Omcfull$ be two \SROIQ ontologies that share the same vocabulary $N_C$, $N_R$, and $N_I$. The \emph{upward cover} and \emph{downward cover} for a concept $C$ are given by
  \begin{alignat*}{2}
    \UpC_{\Omcref,\Omcfull}(C) &= \{ &&D \in \sub(\Omcfull) \mid C \sqsubseteq_\Omcref D \text{ and } \\
    &&&\nexists D' \in \sub(\Omcfull) \text{ with } C \sqsubset_\Omcref D' \sqsubset_\Omcref D \} \enspace, \\
    \DownC_{\Omcref,\Omcfull}(C) &= \{ &&D \in \sub(\Omcfull) \mid D \sqsubseteq_\Omcref C \text{ and } \\
    &&&\nexists D' \in \sub(\Omcfull) \text{ with } D \sqsubset_\Omcref D' \sqsubset_\Omcref C \} \enspace.
  \end{alignat*}

  The upward and downward covers for a role $R$ are given by
  \begin{alignat*}{2}
    \UpC_{\Omcref,\Omcfull}(R) &= \{ &&S \in \Lmc(N_R) \mid R \sqsubseteq_\Omcref S \text{ and } \\
    &&& \nexists S' \in \Lmc(N_R) \text{ with } R \sqsubset_\Omcref S' \sqsubset_\Omcref S \text{ and } \\
    &&& S, S' \text{ are simple in } \Omcfull \} \enspace, \\
    \DownC_{\Omcref,\Omcfull}(R) &= \{ &&S \in \Lmc(N_R) \mid S \sqsubseteq_\Omcref R \text{ and } \\
    &&&\nexists S' \in \Lmc(N_R) \text{ with } S \sqsubset_\Omcref S' \sqsubset_\Omcref R \text{ and } \\
    &&& S, S' \text{ are simple in } \Omcfull \} \enspace.
  \end{alignat*}

  The upward and downward covers for a non-negative integer $n$ are given by
  \begin{alignat*}{2}
    \UpC_{\Omcref,\Omcfull}(n) &= \{ n, n + 1 \} \enspace, \\
    \DownC_{\Omcref,\Omcfull}(R) &=
    \begin{cases*}
      \{ n \} & \text{ if } n = 0 \\
      \{ n, n - 1 \} & \text{ if } n > 0
    \end{cases*} \enspace.
  \end{alignat*}
\end{definition}

Since they operate only over the subconcepts of $\Omcfull$, on their own, the upward and downward covers of concepts are missing some interesting refinements.
\begin{example} \label{exa:up-cover}
  Let $N_C = \{A, B, C\}$, $N_R = \{ r, s \}$, and $\Omc = \{A \sqsubseteq B, r \sqsubseteq s\}$. $\sub(\Omc) = \{\top, \bot, A, B\}$. The upward cover of $C \sqcup A$ is equal to $\UpC_{\Omc,\Omc}(C \sqcup A) = \{ \top \}$. The potentiality refinement to $C \sqcup B$ will be missed even by iterated application of the upward cover because $C \sqcup B \not\in \sub(\Omc)$. Similarly, $\UpC_{\Omc,\Omc}(\forall r.A) = \{ \top \}$, even if \  $\forall r.B$ and $\forall s.A$ are reasonable generalizations.
\end{example}

% define the abstract refinement operator
% define the concrete generalization and specialization operator
To also capture these omissions, we define generalization and specialization operators that exploit the recursive structure of the concept being refined to generate more complex refinements. For convenience, we also define these operators for roles.

\begin{definition}
  Let $\refine$ and $\corefine$ be two functions with domain $\Lmc(N_C, N_R, N_I) \cup \Lmc(N_C) \cup \mathbb{N}_0$. They map every concept to a finite subset of $\Lmc(N_C, N_R, N_I)$, every role to a subset of $\Lmc(N_C)$, and every non-negative integer to a finite subset of $\mathbb{N}_0$.
  The \emph{abstract refinement operator} is defined recursively by induction on the structure of concepts as follows.
  \begin{alignat*}{2}
    \zeta_{\refine,\corefine}(A) &= &&\refine(A) \quad, A \in N_C \cup \{\top, \bot\} \enspace, \\
    \zeta_{\refine,\corefine}(\lnot C) &= &&\refine(\lnot C) \cup \{ \lnot C' \mid C' \in \zeta_{\corefine,\refine}(C) \} \enspace, \\
    \zeta_{\refine,\corefine}(C \sqcap D) &= &&\refine(C \sqcap D) \cup \{ C' \sqcap D \mid C' \in \zeta_{\refine,\corefine}(C) \}
    \cup \{ C \sqcap D' \mid D' \in \zeta_{\refine,\corefine}(D) \} \enspace, \\
    \zeta_{\refine,\corefine}(C \sqcup D) &= &&\refine(C \sqcup D) \cup \{ C' \sqcup D \mid C' \in \zeta_{\refine,\corefine}(C) \}
    \cup \{ C \sqcup D' \mid D' \in \zeta_{\refine,\corefine}(D) \} \enspace, \\
    \zeta_{\refine,\corefine}(\forall R.C) &= &&\refine(\forall R.C) \cup \{ \forall R'.C \mid R' \in \corefine(R) \}
    \cup \{ \forall R.C' \mid C' \in \zeta_{\refine,\corefine}(C) \} \enspace, \\
    \zeta_{\refine,\corefine}(\exists R.C) &= &&\refine(\exists R.C) \cup \{ \exists R'.C \mid R' \in \refine(R) \}
    \cup \{ \exists R.C' \mid C' \in \zeta_{\refine,\corefine}(C) \} \enspace, \\
    &&& \text{\SROIQ concepts:} \\
    \zeta_{\refine,\corefine}(\nominal i) &= &&\refine(\nominal i) \enspace, \\
    \zeta_{\refine,\corefine}(\exists R.\self) &= &&\refine(\exists R.\self) \cup \{ \exists R'.\self \mid R' \in \refine(R) \} \enspace, \\
    \zeta_{\refine,\corefine}(\more n R.C) &= &&\refine(\more n R.C) \cup \{ \more n R'.C \mid R' \in \refine(R) \} \\
    &&& \cup \{ \more n R.C' \mid C' \in \zeta_{\refine,\corefine}(C) \}
    \cup \{ \more n' R.C \mid n' \in \corefine(C) \} \enspace, \\
    \zeta_{\refine,\corefine}(\less n R.C) &= &&\refine(\less n R.C) \cup \{ \less n R'.C \mid R' \in \corefine(R) \} \\
    &&& \cup \{ \less n R.C' \mid C' \in \zeta_{\corefine,\refine}(C) \}
    \cup \{ \less n' R.C \mid n' \in \refine(C) \} \enspace, \\
    &&& \text{\SROIQ roles:} \\
    \zeta_{\refine,\corefine}(R) &= &&\refine(R) \enspace.
  \end{alignat*}

  From the abstract refinement operator $\zeta_{\refine,\corefine}$, two concrete refinement operators, the \emph{generalization operator} and \emph{specialization operator} are respectively defined as
  \begin{align*}
    \gamma_{\Omcref,\Omcfull} &= \zeta_{\UpC_{\Omcref,\Omcfull},\DownC_{\Omcref,\Omcfull}} \enspace \text{and} \\
    \rho_{\Omcref,\Omcfull} &= \zeta_{\DownC_{\Omcref,\Omcfull},\UpC_{\Omcref,\Omcfull}} \enspace.
  \end{align*}
\end{definition}

Revisiting the case in \cref{exa:up-cover} we observe that $\gamma_{\Omc,\Omc}(C \sqcup A) = \{ \top, \top \sqcup A, C \sqcup A, C \sqcup B \}$ does contain $C \sqcup B$ as a possible refinement. Similarly, $\gamma_{\Omc,\Omc}(\forall r.A) = \{ \top, \forall r.A, \forall s.A, \forall r.B \}$ contains $\forall r.B$. We will show now some basic properties of $\gamma_{\Omcref,\Omcfull}$ and $\rho_{\Omcref,\Omcfull}$ that will prove useful in the remainder of this paper.

\begin{lemma}\label{lem:basic}
  For every pair of \SROIQ ontologies $\Omcref, \Omcfull$ and every pair of concepts or roles $X, Y \in \Lmc(N_C, N_R, N_I) \cup \Lmc(N_R)$:
  \newcommand\litem[1]{\item{\bfseries #1:\enspace }}
  \begin{enumerate}
    \litem{generalisation}\label{lem:generalisation} if $X \in \gamma_{\Omcref,\Omcfull}(Y)$ then $Y \sqsubseteq_{\Omcref} X$ \\
    \textbf{specialisation:\enspace} if $X \in \rho_{\Omcref,\Omcfull}(Y)$ then $X \sqsubseteq_{\Omcref} Y$
    \litem{generalisation finiteness} $\gamma_{\Omcref,\Omcfull}(X)$ is finite \\
    \textbf{specialisation finiteness:\enspace} $\rho_{\Omcref,\Omcfull}(X)$ is finite
  \end{enumerate}
\end{lemma}

% define the weakening operator
We define now the \emph{axiom weakening operator} using these generalization and specialization operators.

\begin{definition}
  Given an axiom $\phi$, the set of \emph{weakenings} with respect to the reference ontology $\Omcref$ and full ontology $\Omcfull$, written $g_{\Omcref,\Omcfull}(\phi)$ is defined such that
  \begin{alignat*}{2}
    g_{\Omcref,\Omcfull}(C \sqsubseteq D) &={} &&\{ C' \sqsubseteq D \mid C' \in \rho_{\Omcref,\Omcfull} (C) \} \cup \{ C \sqsubseteq D' \mid D' \in \gamma_{\Omcref,\Omcfull}(D) \} \enspace, \\
    g_{\Omcref,\Omcfull}(C(a)) &={} && \{ C'(a) \mid C' \in \gamma_{\Omcref,\Omcfull}(C) \} \enspace, \\
    g_{\Omcref,\Omcfull}(R(a, b)) &={} && \{ R'(a, b) \mid R' \in \gamma_{\Omcref,\Omcfull}(R) \} \cup \{ R(a, b), \bot \sqsubseteq \top \} \enspace, \\
    g_{\Omcref,\Omcfull}(\lnot R(a, b)) &={} && \{ \lnot R'(a, b) \mid R' \in \rho_{\Omcref,\Omcfull}(R) \} \cup \{ \lnot R(a, b), \bot \sqsubseteq \top \} \enspace, \\
    g_{\Omcref,\Omcfull}(a = b) &={} && \{ a = b, \bot \sqsubseteq \top \} \enspace,
    \quad g_{\Omcref,\Omcfull}(a \not= b) = \{ a \not= b, \bot \sqsubseteq \top \} \enspace, \\
    &&& \text{\SROIQ axioms:} \\
    g_{\Omcref,\Omcfull}(\disjoint(R, S)) &={} &&\{ \disjoint(R', S) \mid R' \in \rho_{\Omcref,\Omcfull} (R) \} \\
    &&& \cup \{ \disjoint(R, S') \mid S' \in \rho_{\Omcref,\Omcfull}(S) \} \\
    &&& \cup \{ \disjoint (R, S), \bot \sqsubseteq \top \} \enspace, \\
    g_{\Omcref,\Omcfull}(S_1 \circ \cdots \circ S_n \sqsubseteq R) &={} && \{ S_1 \circ \cdots \circ S_i' \circ \cdots \circ S_n \sqsubseteq R \mid S_i' \in \rho_{\Omcref,\Omcfull}(S_i) \text{ for } i = 1, \dots, n \} \\
    &&& \cup \{ S_1 \sqsubseteq R' \mid R' \in \gamma_{\Omcref,\Omcfull} \text{ and } n = 1 \text{ and } S_1 \text{ is simple in } \Omcfull \} \\
    &&& \cup \{ S_1 \circ \cdots \circ S_n \sqsubseteq R, \bot \sqsubseteq \top \} \enspace.
  \end{alignat*}
\end{definition}

% \begin{example}
% % give an example of the weakening operator
% \end{example}

The axioms in the set $g_{\Omcref,\Omcfull}(\phi)$ are indeed weaker than $\phi$ for every axiom $\phi$, in the sense that, given the reference ontology $\Omcref$, $\phi$ entails them and the opposite in not necessarily true.

\begin{lemma} \label{lem:weaker}
  For every \SROIQ axiom $\phi$, if $\phi' \in g_{\Omcref,\Omcfull}(\phi)$, then $\phi \models_\Omcref \phi'$
\end{lemma}

\begin{proof} We will handle each type of axiom separately.
  \begin{itemize}
    \item If $\phi = C \sqsubseteq D$, suppose $\phi' = C' \sqsubseteq D'$. From \cref{lem:basic}.\ref{lem:generalisation} we know that $C' \sqsubseteq_\Omcref C$ and $D \sqsubseteq_\Omcref D'$. By transitivity of subsumption, we conclude that $C \sqsubseteq D \models_\Omcref C' \sqsubseteq D'$.
    \item If $\phi = C(a)$, suppose $\phi' = C'(a)$. From \cref{lem:basic}.\ref{lem:generalisation} we know that $C \sqsubseteq_\Omcref C'$. Given any model $I$ of $\Omcref \cup \{ \phi \}$, $a^I \in C^I$. Since $C^I \subseteq C'^I$ in every model of $\Omcref$, $a^I \in C'^I$. We conclude that $C(a) \models_\Omcref C'(a)$.
    \item If $\phi = R(a, b)$, suppose $\phi' = R'(a, b)$. From \cref{lem:basic}.\ref{lem:generalisation} we know that $R \sqsubseteq_\Omcref R'$. Given any model $I$ of $\Omcref \cup \{ \phi \}$, $\langle a^I, b^I \rangle \in R^I$. Since $R^I \subseteq R'^I$ in every model of $\Omcref$, $\langle a^I, b^I \rangle \in R'^I$. We conclude that $R(a, b) \models_\Omcref R'(a, b)$.
    \item If $\phi = \lnot R(a, b)$, suppose $\phi' = R'(a, b)$. From \cref{lem:basic}.\ref{lem:generalisation} we know that $R' \sqsubseteq_\Omcref R$. Given any model $I$ of $\Omcref \cup \{ \phi \}$, $\langle a^I, b^I \rangle \not\in R^I$. Since $R'^I \subseteq R^I$ in every model of $\Omcref$, $\langle a^I, b^I \rangle \not\in R'^I$. We conclude that $\lnot R(a, b) \models_\Omcref \lnot R'(a, b)$.
    \item If $\phi = \disjoint(R, S)$, suppose $\phi' = \disjoint(R', S')$. From \cref{lem:basic}.\ref{lem:generalisation} we know that $R' \sqsubseteq_\Omcref R$ and $S' \sqsubseteq_\Omcref S$. Given any model $I$ of $\Omcref \cup \{ \phi \}$, $R^I \cap S^I = \emptyset$. Since $R'^I \subseteq R^I$ and $S'^I \subseteq S^I$ in every model of $\Omcref$, $R'^I \cap S'^I = \emptyset$. We conclude that $\disjoint(R, S) \models_\Omcref \disjoint(R', S')$.
    \item If $\phi = S_1 \circ \cdots \circ S_n \sqsubseteq R$, suppose $\phi' = S_1' \circ \cdots \circ S_n' \sqsubseteq R'$. From \cref{lem:basic}.\ref{lem:generalisation} we know that $R \sqsubseteq_\Omcref R'$ and $S_i' \sqsubseteq_\Omcref S_i$ for $i = 1, \dots, n$. Given any model $I$ of $\Omcref \cup \{ \phi \}$, $S_1^I \circ \cdots \circ S_n^I \subseteq R^I$. Since $R^I \subseteq R'^I$ and $S_i'^I \subseteq S_i^I$ for $i = 1, \dots, n$ in every model of $\Omcref$, $S_1'^I \circ \cdots \circ S_n'^I \subseteq R'^I$. We conclude that $S_1 \circ \cdots \circ S_n \sqsubseteq R \models_\Omcref S_1' \circ \cdots \circ S_n' \sqsubseteq R'$.
  \end{itemize}
\end{proof}

Clearly, replacing an axiom in the full ontology with a weakening can not reduce the number of  models of the ontology. However, for the weakening to be useful in practice, we must show additionally that by adding the weakened axioms to the ontology will not violate any of the constraints that ensure the decidability of \SROIQ. To do this, we show first that all roles that are simple in $\Omcfull$ are also simple in the ontology obtained by adding the weakening of any axiom.

% proof that the constrains are retained
\begin{lemma} \label{lem:simple-roles}
  For every axiom $\phi \in \Omcfull$ and role $R$, if $\phi' \in g_{\Omcref,\Omcfull}(\phi)$ and $R$ simple in $\Omcfull$, then $R$ is simple in $\Omcfull \cup \{ \phi' \}$.
\end{lemma}

\begin{proof}(\emph{Sketch})
  Assume, by contradiction, that $R$ is a simple role in $\Omcfull$ and non-simple in $\Omcfull \cup \{ \phi' \}$. Since $R$ is simple in $\Omcfull$ it is neither the universal nor the existential role, does not appear as the super role in any complex RIA of $\Omcfull$, and neither on the right-hand side of a simple RIA in $\Omcfull$ where the sub role is non-simple. We conclude that $\phi'$ must be a RIA, that has $R$ as the super role and is either complex, or for which the sub role is non-simple. If $\phi'$ is a complex RIA then, by definition of the weakening operator, $\phi$ must be a complex RIA and $R$ must be the super role in $\phi$, making it non-simple in $\Omcfull$, which contradicts our assumption. Similarly, if $\phi'$ is a simple RIA with a non-simple role as the sub role, the sub role of $\phi$ must be equal to that of $\phi'$ because the refinement operators return only roles simple in $\Omcfull$. Further, since the super role of a RIA is only refined if the sub role is simple, $\phi' = \phi$, which means that $R$ is non-simple in $\Omcfull$, which contradicts the assumptions. It follows that such a role $R$ does not exist.
\end{proof}

Note that the proof works also for weakening axioms in an ontology $\Omc$ as long as all non-simple roles in $\Omc$ are also non-simple in $\Omcfull$ (or, equivalently, that all simple roles in $\Omcfull$ are also simple in $\Omc$). This is an important observation, since it means that repeatedly adding weakened axioms is possible. We will show next that the 

\begin{lemma} \label{lem:regularity}
  For every axiom $\phi \in \Omcfull$, if $\phi' \in g_{\Omcref,\Omcfull}(\phi)$ and the role hierarchy of $\Omcfull$ is regular, then the role hierarchy of $\Omcfull \cup \{ \phi' \}$ is also regular.
\end{lemma}

\begin{proof}(\emph{Sketch}) \phantomsection\label{proof:regularity}
  Let us first argue that if there exists a preorder $\preceq$ that satisfies the constraints necessary for checking regularity, then there exists on such that $S_1 \preceq S_2$, $S \preceq R$ and $R \not\preceq S$ for all simple roles $S, S_1, S_2$ and non-simple roles $R$. Firstly, $S_1 \not\preceq S_2$ and $S \not\preceq R$ can not be required, because absence of a tuple is only required for complex RIAs, where the super role must not be a predecessor of the roles on the left-hand side. Since $S_1$ and $S$ are simple, they do not appear as the super role in a complex RIA. Similarly, $R \preceq S$ can not be required. Since $S$ is simple and $R$ non-simple, it can not be required directly through an axiom of the form $R \sqsubseteq S$. By induction, it can not be required through transitivity, since $R \preceq T$ and $T \preceq S$ would have to be required. If $T$ is simple, $R \preceq T$ can not be required, and if $T$ is non-simple, $T \preceq S$ can not be required.

  Since $\Omcfull$ has a regular role hierarchy, there exists such a $\preceq$ for $\Omcfull$. We will show that $\preceq$ is also a witness for regularity of $\Omcfull \cup \{ \phi' \}$. All RIA in $\Omcfull$ are of one of allowed forms for $\preceq$. It is therefore sufficient to verify that $\phi'$ has one of the allowed forms.
  If $\phi' = \phi$ or $\phi'$ is not a RIA, it does not affect the regularity.
  Otherwise, if $\phi'$ is a simple RIA $S \sqsubseteq R$, then by definition of the weakening operator, $S$ is simple in $\Omcfull$. Given that $S$ is simple, $S \preceq R$ holds for simple and non-simple $R$ by our choice of $\preceq$.
  If $\phi'$ is a complex RIA $S_1' \circ \cdots \circ S_n' \sqsubseteq R$, then $\phi$ is also a complex RIA $S_1 \circ \cdots \circ S_n \sqsubseteq R$ and $R$ is non-simple in $\Omcfull$. If $S_i \preceq R$ and $S_i \not\preceq R$, then so will $S_i' \preceq R$ and $R \not\preceq S_i'$, either because $S_i = S_i'$ or because $S_i'$ is simple and $R$ is non-simple. Since $\Omcfull$ has a regular role hierarchy, the only case in which $R \preceq S_i$ is if $S_i = R$. In this case, $S_i' \preceq R$ and $R \not\preceq S_i'$ will still hold if $S_i \not= R$. If $S_i = R$, either $i = 0$ or $i = n$ which is allowed. The only delicate case is if $\phi = R \circ R \sqsubseteq R$, which will result in either $\phi' = S_1' \circ R \sqsubseteq R$ or $R \circ S_2' \sqsubseteq R$, both of which are valid.
\end{proof}

Like for simple roles, also regularity is maintained by repeated addition of weaker axioms. With the help of \cref{lem:simple-roles} and \cref{lem:regularity} we will now sketch a proof showing that adding weakened axioms to a \SROIQ ontology will yield another valid \SROIQ ontology.

\begin{lemma} \label{lem:global-constraints}
  Given that $\Omcref$ and $\Omcfull$ are valid \SROIQ ontologies. For every axiom $\phi \in \Omcfull$, if $\phi' \in g_{\Omcref,\Omcfull}(\phi)$, then $\Omcfull \cup \{ \phi' \}$ is a valid \SROIQ ontology.
\end{lemma}

\begin{proof}(\emph{Sketch})
  We have established already in \cref{lem:regularity}, that the regularity of the RBox will be preserved.
  It is guaranteed by \cref{lem:simple-roles} that all roles that were simple before addition, are still simple afterwards. Therefore, all usages of roles in axioms and concepts that were not touched by the refinement do not pose a problem. The condition static that the upcover and downcover of a role contain only roles that are simple in $\Omcfull$ (and therefore by \cref{lem:simple-roles} also in $\Omcfull \cup \{ \phi' \}$) forces that every refinement of a role is simple. This restriction to simple roles guarantees that no non-simple role may be used in disjoint role axioms, or the scope of cardinality and self constraints.
\end{proof}

\section{Implementing Axiom Weakening for \SROIQ}

% explain how the weakening is used for repair
Refinement operator and axiom weakening have previously been implemented for \ALC in \cite{troquard2018repairing}. Based on this, we have now extended the implementation to cover the full range of \SROIQ axioms and concepts.\footnote{The source code for the implementation is available at \url{https://github.com/rolandbernard/ontologyutils}} The concept refinement and axiom weakening operators for \SROIQ have been implemented as discussed above. Further, we implemented a repair algorithm using the axiom weakening operator based on the procedures already proposed in \cite{troquard2018repairing} and \cite{confalonieri2020towards}. The implementation performs weakening in OWL 2 DL \cite{motik2009owl} and is implemented in Java using the OWL API \cite{horridge2011owl}. A plug-in for the ontology development tool Protégé has also been implemented, but will not be discussed in detail in this paper.\footnote{The Protégé plugin is available at \url{https://github.com/rolandbernard/protege-weakening}} The plug-in allows for manually weakening axioms and executing the automatic repair algorithm.

\begin{algorithm}[t]
  \begin{algorithmic}
    \State $\Omcfull \gets \Omc$
    \State $\Omcref \gets \textnormal{FindMaximalConsistentSubset(\Omc)}$
    \While{$\Omc$ is inconsistent}
      \State $\phi_\textnormal{bad} \gets \textnormal{FindBadAxiom}(\Omc)$
      \State $\Phi_\textnormal{weaker} \gets g_{\Omcref,\Omcfull}(\phi_\textnormal{bad})$
      \State $\phi_\textnormal{weaker} \gets \textnormal{SelectWeakerAxiom}(\Phi_\textnormal{weaker})$
      \State $\Ontology \gets \Ontology \setminus \{\phi_\textnormal{bad}\} \cup \{\phi_\textnormal{weaker}\}$
    \EndWhile
    \State Return $\Omc$
  \end{algorithmic}
  \caption{RepairOntologyWeaken($\Omc$)}
	\label{algo:repair-weaken}
\end{algorithm}

% explain how the reference ontology is selected
% explain how bad axioms are selected
The automatic repair by weakening is implemented as shown in \cref{algo:repair-weaken}. The reference ontology is selected by choosing a maximal consistent subset of the inconsistent ontology. In our implementation used for the evaluation in this paper, the reference ontology was selected by randomly sampling a maximal consistent subset. The procedure $\textnormal{FindBadAxiom}(\Omc)$ may be implemented in a number of ways. Here we consider an implementation that samples some (or all) of the minimal inconsistent subsets of $\Omc$ and selects as the bad axiom the one occurring most frequently. Then, $\textnormal{SelectWeakerAxiom}(\Phi_\textnormal{weaker})$ has been chosen such that is selects an axiom uniformly at random form $\Phi_\textnormal{weaker}$.


\section{Weakening makes you strong: evaluation aspects}

% explain how different repairs are compared
To experimentally evaluate the proposed axiom weakening operator in the context of its use in automatic repair of ontologies, we need some way to compare the quality of repair. As has already been discussed in \cite{troquard2018repairing}, the problem of deciding which of two possible repaired ontologies $\Omc_1$ or $\Omc_2$ is preferable is not generally well-defined. Similar to what has been proposed in \cite{troquard2018repairing} we will base the evaluation of the repairs on the size of the \emph{inferred class hierarchy}. To compare two possible repairs, we use the \emph{inferable information content} as defined in \cite{troquard2018repairing}. Some weakenesses of this measure when it comes to evaluating repairs, like the fact that only atomic concepts are considered, have already been discussed in \cite{troquard2018repairing}. For the case of repairing \SROIQ ontologies this is even more relevant, since the role hierachy is compleaty ignored.

\begin{definition}
  The \emph{inferred class hierarchy} of an ontology $\Omc$ is given by
  \begin{align*}
    \Inf(\Omc) = \{ A \sqsubseteq B \mid A, B \in N_C \text{ and } \Omc \models A \sqsubseteq B \} \enspace.
  \end{align*}

  The \emph{inferable information content} of an ontology $\Omc_1$ with respect to another ontology $\Omc_2$ is given by
  \begin{align*}
    \qual(\Omc_1, \Omc_2) = \frac{\mathbf{card}(\Inf(\Omc_1) \setminus \Inf(\Omc_2))}{\mathbf{card}(\Inf(\Omc_1) \setminus \Inf(\Omc_2)) + \mathbf{card}(\Inf(\Omc_2) \setminus \Inf(\Omc_1))} \enspace,
  \end{align*}
  where $\mathbf{card}(X)$ is the cardinality of the set $X$.
\end{definition}


% explain which ontologies where chosen and from where (BioPortal)
For the experimental evaluation we have selected ontologies of varying size and expressivity from BioPortal\footnote{\url{https://bioportal.bioontology.org/}}. Additionally the pizza ontology\footnote{Available from Protégé at \url{https://protege.stanford.edu/ontologies/pizza/pizza.owl}} (cite or footnote) was included in the testing.

\begin{table}
  \centering
  \begin{tabular}{|l l l r r r r|}
    \hline
    Abbreviation & Name & Expressivity & Axioms & Concepts & Roles & Subconcepts \\
    \hline
    admin & Nurse Administrator & $\mathcal{ALCHOIF}$ & 229 & 42 & 29 & 144 \\
    ahso & Animal Health Surveillance Ontology & $\mathcal{ALCRIF}$ & 166 & 38 & 31 & 104 \\
    cdao & Comparative Data Analysis Ontology & $\mathcal{ALCROIQ}$ & & & & \\
    cdpeo & Chronic Disease Patient Education Ontology & $\mathcal{ALCHF}$ & & & & \\
    covid19-ibo & Covid19 Impact on Banking Ontology & $\mathcal{ALCH}$ & & & & \\
    ecp & Electronic Care Plan & $\mathcal{ALCRQ}$ & & & & \\
    emo & Enzyme Mechanism Ontology & $\mathcal{ALCHQ}$ & & & & \\
    evi & Evidence Graph Ontology & $\mathcal{ALCRI}$ & & & & \\
    falls & Falls Prevention & $\mathcal{ALCH}$ & & & & \\
    fo Fern Ontology & $\mathcal{ALCHI}$ & & & & \\
    gbm & Glioblastoma & $\mathcal{ALCIF}$ & & & & \\
    gfvo & Genomic Feature and Variation Ontology & $\mathcal{ALCH}$ & & & & \\
    koro & Knowledge Object Reference Ontology & $\mathcal{ALCHI}$ & & & & \\
    lico & Liver Case Ontology & $\mathcal{ALCHQ}$ & & & & \\
    mamo & Mathematical Modelling Ontology & $\mathcal{ALCR}$ & & & & \\
    mpio & Minimum PDDI Information Ontology & $\mathcal{ALCH}$ & & & & \\
    provo & Provenance Ontology & $\mathcal{ALCRIN}$ & & & & \\
    qudt & Quantities, Units, Dimensions, and Types Ontology & $\mathcal{SHOIQ}$ & & & & \\
    trans & Nurse Transitional & $\mathcal{ALCROIF}$ & & & & \\
    triage & Nurse triage & $\mathcal{ALCHF}$ & & & & \\
    vio & Vaccine Investigation Ontology & $\mathcal{ALCRI}$ & & & & \\
    \hline
    pizza & Pizza Ontology & $\mathcal{SHOIN}$ & & & & \\
    \hline
  \end{tabular}
  \caption{The BioPortal ontologies used for evaluation. The number of axioms, concepts, roles, and subconcepts are taken after preprocessing.}
\end{table}

% explain how inconsistent ontologies were generated from them and then repaired
\begin{algorithm}[t]
  \begin{algorithmic}
    \While{$\Omc$ is inconsistent}
      \State $\phi_\textnormal{bad} \gets \textnormal{FindBadAxiom}(\Omc)$
      \State $\Ontology \gets \Ontology \setminus \{\phi_\textnormal{bad}\}$
    \EndWhile
    \State Return $\Omc$
  \end{algorithmic}
  \caption{RepairOntologyRemove($\Omc$)}
	\label{algo:repair-remove}
\end{algorithm}

% show results of the evaluation

\section{Outlook}

% better measures for comparing ontologies
% better heuristics to steer the algorithm
% complex roles in up and down covers
% more permissive refinement of role inclusions
% working directly with OWL 2 axioms

%%
%% Define the bibliography file to be used
\bibliography{biblio}

%%
%% If your work has an appendix, this is the place to put it.
\appendix


\end{document}

%%
%% End of file
