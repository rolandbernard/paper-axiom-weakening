%% The first command in your LaTeX source must be the \documentclass command.
%%
%% Options:
%% twocolumn : Two column layout.
%% hf: enable header and footer.
\documentclass[
% twocolumn,
% hf,
]{ceurart}

%%
%% One can fix some overfulls
\sloppy

%%
%% Minted listings support 
%% Need pygment <http://pygments.org/> <http://pypi.python.org/pypi/Pygments>
\usepackage{listings}
%% auto break lines
\lstset{breaklines=true}

%%
%% DL Logo for inline use
\usepackage{graphbox}
\DeclareRobustCommand{\DLLogo}{%
  \begingroup\normalfont
  \kern-1.75pt\includegraphics[align=c,height=1.25\baselineskip]{dl}\kern-1.5pt%
  \endgroup
}

%%
%% AMS Theorems
\usepackage{amsthm}
\newtheorem{theorem}{Theorem}
\newtheorem{definition}{Definition}
\newtheorem{example}{Example}

%%%%%%%%%%%%%%%%%%%%%%%%%%%%%%%%%%%%%%%%
%%% OUR SETTINGS
\usepackage{amsmath}
\usepackage{amssymb}
\usepackage{mathtools}
\usepackage{algorithm}
\usepackage{algpseudocode}
% \usepackage{theorem}
\usepackage{xspace}
\usepackage{dl}
\usepackage{url}
%\usepackage[numbers]{natbib}
\clubpenalty = 10000
\widowpenalty = 10000
\displaywidowpenalty = 10000
% theorems and the like
% \newtheorem{theorem}{Theorem}
\newtheorem{lemma}{Lemma}
% \newtheorem{proposition}{Proposition}
% \newtheorem{corollary}{Corollary}

% \theorembodyfont{\rmfamily}
% \newtheorem{definition}{Definition}
% \newtheorem{example}{Example}

% \theorembodyfont{\slshape}
% \newtheorem{assumption}{Assumption}

% \theorembodyfont{\slshape}
% \newtheorem{hypothesis}{Hypothesis}

% \newenvironment{justification}{\textsf{Justification:}}{\hfill $\Box$}
%\newenvironment{proof}{\noindent\textsc{proof:}}{\hfill $\square$ \bigskip}


\newcommand{\AL}{\ensuremath{\mathcal{AL}}\xspace}
\newcommand{\ALC}{\ensuremath{\mathcal{ALC}}\xspace}
\newcommand{\SROIQ}{\ensuremath{\mathcal{SROIQ}}\xspace}
\newcommand{\onto}[1]{\ensuremath{\mathsf{#1}}}
\usepackage[colorinlistoftodos]{todonotes}

\newcommand{\cg}
{\ensuremath{\blacktriangle}\xspace}
%{\ensuremath{\dot{\sqcup}}\xspace}

\newcommand{\todoT}[1]{\todo[fancyline,size=\small,color=orange!40]{\textbf{TBD:} #1}\xspace}
\newcommand{\todoR}[1]{\todo[fancyline,size=\small,color=orange!40]{\textbf{rc:} #1}\xspace}

\newcommand{\todoO}[1]{\todo[fancyline,size=\small,color=red!20]{\textbf{ok:} #1}\xspace}

\newcommand{\tododo}[1]{\todo[fancyline,size=\small,color=red!60]{\textbf{ToDO:} #1}\xspace}
\newcommand{\todoin}[1]{\todo[inline,size=\small,color=green!20]{\textbf{HERE:} #1}\xspace}


%\usepackage{pgf}https://preview.overleaf.com/public/pqghhhhwjqpw/images/986e3685291cb332a5fbe5a979b5045d2e649a43.jpeg
\usepackage{tikz}
%\usetikzlibrary{arrows,automata}
\usetikzlibrary{positioning,calc,backgrounds}

%%%%%%%MACROS%%%%%%%%
\newcommand{\Lmc}{\ensuremath{\mathcal{L}}\xspace}
\newcommand{\Imc}{\ensuremath{\mathcal{I}}\xspace}
\newcommand{\Jmc}{\ensuremath{\mathcal{J}}\xspace}
\newcommand{\Tmc}{\ensuremath{\mathcal{T}}\xspace}
\newcommand{\Amc}{\ensuremath{\mathcal{A}}\xspace}
\newcommand{\Rmc}{\ensuremath{\mathcal{R}}\xspace}
\newcommand{\Omc}{\ensuremath{\mathcal{O}}\xspace}
\newcommand{\Omcref}{\ensuremath{{\mathcal{O}^\textnormal{ref}}}\xspace}
\newcommand{\Omcfull}{\ensuremath{{\mathcal{O}^\textnormal{full}}}\xspace}
\newcommand{\Ontology}{\Omc}

\newcommand{\EL}{\ensuremath{\mathcal{E\!L}}\xspace}
\newcommand{\elpp}{\ensuremath{\mathcal{E\!L}^{++}}\xspace}

\newcommand{\Inf}{{\ensuremath{\mathsf{Inf}}}\xspace}
\newcommand{\qual}{{\ensuremath{\mathsf{IIC}}}\xspace}

\newcommand{\UpC}{{\ensuremath{\mathsf{UpCover}}}\xspace}
\newcommand{\DownC}{{\ensuremath{\mathsf{DownCover}}}\xspace}

\newcommand{\disjoint}{\ensuremath{\mathit{disjoint}}\xspace}
\newcommand{\self}{\ensuremath{\mathit{Self}}\xspace}
\newcommand{\less}[2]{\ensuremath{\leq #1~#2}\xspace}
\newcommand{\more}[2]{\ensuremath{\geq #1~#2}\xspace}
\newcommand{\nominal}[1]{\ensuremath{\{#1\}}\xspace}

\newcommand{\inv}{\ensuremath{\mathit{inv}}\xspace}
\newcommand{\refine}{\ensuremath{\mathop{\uparrow}}\xspace}
\newcommand{\corefine}{\ensuremath{\mathop{\downarrow}}\xspace}
%\DeclareMathOperator*{\argmax}{\mathsf{arg\,max}}
\DeclareMathOperator*{\argmax}{\mathsf{argmax}}

%%%%%%%%%%%%%%%%%%%%%%%%%%%%%%%%%%%%%%%%

%%
%% end of the preamble, start of the body of the document source.
\begin{document}

%%
%% Rights management information.
%% CC-BY is default license.
\copyrightyear{2023}
\copyrightclause{Copyright for this paper by its authors.
  Use permitted under Creative Commons License Attribution 4.0
  International (CC BY 4.0).}

%%
%% This command is for the conference information
\conference{\DLLogo{} DL 2023: 36th International Workshop on Description Logics,
  September 2--4, 2023, Rhodes, Greece}

%%
%% The "title" command
\title{Implementing Axiom Weakening for SROIQ}

%%
%% The "author" command and its associated commands are used to define
%% the authors and their affiliations.

\author[1]{Roland Bernard}[
email=roland.bernard@student.unibz.it,
]
\author[1]{Oliver Kutz}[
email=oliver.kutz@unibz.it,
]
\author[1]{Nicolas Troquard}[
email=nicolas.troquard@unibz.it,
]
\address[1]{
Free University of Bozen-Bolzano, Italy
}

%%
%% The abstract is a short summary of the work to be presented in the
%% article.
\begin{abstract}
Axiom weakening is a technique that allows for a fine-grained repair of inconsistent ontologies. Its main advantage is that it repairs ontologies by making axioms
less restrictive rather than by deleting them, employing refinement operators. In this paper, we build on previously introduced axiom weakening for \ALC, and show how it can be extended to deal with \SROIQ, the expressive and decidable description logic underlying OWL 2 DL.
We here focus on describing a prototype implementation computing axiom weakening for \SROIQ and discuss a number of performance and evaluation aspects.
%the definitions of the refinement operators to deal with \SROIQ constructs, in particular with %such as reflexive and irreflexive roles, disjoint roles, 
%role hierarchies,  cardinality constraints and nominals, and illustrate its application. Finally, we discuss the problem of termination of an iterated weakening procedure.
\end{abstract}

%%
%% Keywords. The author(s) should pick words that accurately describe
%% the work being presented. Separate the keywords with commas.
\begin{keywords}
  Description Logic \sep
  Knowledge refinement \sep
  Prot\'eg\'e
\end{keywords}

%%
%% This command processes the author and affiliation and title
%% information and builds the first part of the formatted document.
\maketitle

\section{Introduction: Weakening for debugging}

% motivate repair using axiom weakening

\begin{example}
% give an example ontology where classical repairs are insufficient
% give an example where it is advantageous to weaken the role hierarchy
\end{example}

% explain other gentle repair approaches
% briefly describe refinement operators
% briefly describe what we have done

\section{Preliminaries}

\section{Axiom Weakening for \ALC}

% describe the ALC syntax and semantics
Formally, an ontology is a set of statements expressed in a suitable logical language and with the purpose of describing a specific domain of interest. 

% define sub concepts

% define upward and downward covers

\begin{example}
% give an example of some cover applications
\end{example}

% define the abstract and concrete refinement operators

\begin{example}
% give an example of the concrete refinement operators
\end{example}

% define the weakening operator

\begin{example}
% give an example of the weakening operator
\end{example}

\section{Extending Weakening to \SROIQ}

% describe the SROIQ syntax
% describe roles and concepts
We now give a brief description of the DL \SROIQ; for full details see \cite{baader_horrocks_lutz_sattler_2017,HorrocksKutzSattlerKR2006}. The syntax of \SROIQ is based on a vocabulary of three disjoint sets $N_I$, $N_R$, $N_C$ of \emph{individual names}, \emph{role names}, and \emph{concept names}. The set of \SROIQ \emph{concepts} and \emph{roles} is generated by the following grammar.

\begin{eqnarray*}
  R, S & ::= & U \mid E \mid r \mid r^{-} \enspace,\\
  C & ::= & \bot \mid \top \mid A \mid \neg C \mid C \sqcap C \mid C \sqcup C \mid \forall R.C \mid \exists R.C \mid \\ 
  & & \more n S.C \mid \less n S.C \mid \exists S.\self \mid \nominal i \enspace,
\end{eqnarray*}

where $r \in N_R$ is a role name, $A \in N_C$ is a concept name, $i \in N_i$ is an individual name and $n \in \mathbb{N}_0$ is a non-negative integer. $U$ and $E$ are respectively the universal role and empty role. $S$ is a \emph{simple role} (see below) in the RBox $\Rmc$.

% describe the TBox, ABox and RBox statements
A \emph{TBox} $\Tmc$ is a finite set of concept inclusions (GCIs) of the form $C \sqsubseteq D$ where $C$ and $D$ are concepts. The TBox is used to stores terminological knowledge concerning the relationship between concepts. A \emph{ABox} $\Amc$ is a finite set of statements of the form $R(a)$, $\lnot R (a)$, $a = b$, and $a \not= b$ where $R$ is a role and $a$ and $b$ are individual names. The ABox expresses knowledge regarding individuals in the domain. A \emph{RBox} $\Rmc$ is a finite set of role inclusions (RIAs) of the form $R_1 \circ \cdots \circ R_n \sqsubseteq R$, and disjoint role axioms $\disjoint(S_1, S_2)$ where $R$, $R_1$, $\dots$, $R_n$, $S_1$, and $S_2$ are roles. $S_1$ and $S_2$ are simple (see next) in the RBox $\Rmc$. The special case of $n = 1$ is a simple role inclusion, while we call the cases where $n > 1$ complex role inclusions. The RBox represents knowledge about the relationship between roles.

% describe simple and complex roles since it is relevant
The set of \emph{non-simple} roles in $\Rmc$ is the smallest set such that: $U$ and $E$ are non-simple; any role $R$ that appears on the right-hand side of a complex role inclusion $R_1 \circ \cdots \circ R_n \sqsubseteq R$ where $n > 1$ is non-simple; any role $R$ that appears on the right-hand side of a simple role inclusion $S \sqsubseteq R$ where $S$ is non-simple, is also non-simple; and a role $r$ is non-simple if and only if $r^-$ is non-simple.
All other roles are \emph{simple}.

% describe regularity since it is relevant
For convenience, let us define the function $\inv(R)$ such that $\inv(r) = r^-$ and $\inv(r^-) = r$ for all role names $r \in N_R$. A RBox $\Rmc$ is \emph{regular} if there exists a pre-order $\preceq$, i.e., a transitive and reflexive relation, over the set of roles such that $R \preceq S \iff \inv(R) \preceq \inv(S)$, $R \preceq S \iff \inv(R) \preceq S$, and all RIAs in $\Rmc$ are of the forms:
$\inv(R) \sqsubseteq R$,
$R \circ R \sqsubseteq R$,
$S \sqsubseteq R$, $R \circ S_1 \circ \cdots \circ S_n \sqsubseteq R$,
$S_1 \circ \cdots \circ S_n \circ R \sqsubseteq R$, or
$S_1 \circ \cdots \circ S_n \sqsubseteq R$
where $r \in N_R$ is a role name and $R$, $S$, $S_1, \cdots, S_n$ are roles such that $S \preceq R$, $S_i \preceq R$, and $R \not\preceq S_i$ for $i = 1, \dots, n$.

A \SROIQ ontology $\Omc = \Tmc \cup \Amc \cup \Rmc$ consists of a TBox $\Tmc$, an ABox $\Amc$, and a RBox $\Rmc$, where $\Rmc$ is regular.

% describe the semantics
% whats a interpretation, when is it a model
The semantics of \SROIQ are defined using \emph{interpretations} $I = \langle \Delta^I, \cdot^I \rangle$ where $\Delta^I$ is a non-empty \emph{domain} and $\cdot^I$ is a function associating to each individual name $a$ an element of the domain $a^I \in \Delta^I$, to each concept $C$ a subset of the domain $C^I \subseteq \Delta^I$, and to each role $R$ a binary relation on the domain $R^I \subseteq \Delta^I \bigtimes \Delta^I$; see \cite{baader_horrocks_lutz_sattler_2017,HorrocksKutzSattlerKR2006} for further details. An interpretation $I$ is a \emph{model} for $\Omc$ if it satisfies all the axioms in $\Omc$.

% describe subsumptions of concepts and roles
Given two concepts $C$ and $D$ we say that $C$ is \emph{subsumed} by $D$ (or $D$ \emph{subsumes} $C$) with respect to the ontology $\Omc$, written $C \sqsubseteq_\Omc D$, if $C^I \subseteq D^I$ in every model $I$ of $\Omc$. Further $C$ is \emph{strictly subsumed by} $D$, written $C \sqsubset_\Omc D$, if $C \sqsubseteq_\Omc D$ but not $D \sqsubseteq_\Omc C$. Analogously, given two roles $R$ and $S$, $R$ is subsumed by $S$ with respect to $\Omc$ ($R \sqsubseteq_\Omc S$) if $R^I \sqsubseteq S^I$ in all models $I$ of $\Omc$. Again, $R \sqsubset_\Omc S$ holds if $R \sqsubseteq_\Omc S$ but not $D \sqsubseteq_\Omc C$

% explain the complications of weakening SROIQ

\begin{example}
% example of the complications of weakening SROIQ
\end{example}

% lay out the constraints that we want to maintain
% explain how each of them can be ensured
% define upward and downward covers
% define the abstract refinement operator
% define the concrete generalization and specialization operator
% define the weakening operator

\begin{example}
% give an example of the weakening operator
\end{example}

% proof that the constrains are retained

\section{Implementing Axiom Weakening for \SROIQ}

% explain how the weakening is used for repair
% explain how the reference ontology is selected
% explain how bad axioms are selected

\section{Weakening makes you strong: evaluation aspects}

% explain how different repairs are compared
% explain how inconsistent ontologies were generated
% explain that the OWL 2 ontologies were normalized to only use SROIQ axioms
% show results of the evaluation

\section{Outlook}

% complex roles in up and down covers
% more permissive refinement of role inclusions
% working directly with OWL 2 axioms
% how to steer the repair process

%%
%% Define the bibliography file to be used
\bibliography{biblio}

%%
%% If your work has an appendix, this is the place to put it.
\appendix


\end{document}

%%
%% End of file
