% theorems and the like
% \newtheorem{theorem}{Theorem}
\newtheorem{lemma}{Lemma}
% \newtheorem{proposition}{Proposition}
% \newtheorem{corollary}{Corollary}

% \theorembodyfont{\rmfamily}
% \newtheorem{definition}{Definition}
% \newtheorem{example}{Example}

% \theorembodyfont{\slshape}
% \newtheorem{assumption}{Assumption}

% \theorembodyfont{\slshape}
% \newtheorem{hypothesis}{Hypothesis}

% \newenvironment{justification}{\textsf{Justification:}}{\hfill $\Box$}
%\newenvironment{proof}{\noindent\textsc{proof:}}{\hfill $\square$ \bigskip}


\newcommand{\AL}{\ensuremath{\mathcal{AL}}\xspace}
\newcommand{\ALC}{\ensuremath{\mathcal{ALC}}\xspace}
\newcommand{\SROIQ}{\ensuremath{\mathcal{SROIQ}}\xspace}
\newcommand{\onto}[1]{\ensuremath{\mathsf{#1}}}
\usepackage[colorinlistoftodos]{todonotes}

\newcommand{\cg}
{\ensuremath{\blacktriangle}\xspace}
%{\ensuremath{\dot{\sqcup}}\xspace}

\newcommand{\todoT}[1]{\todo[fancyline,size=\small,color=orange!40]{\textbf{TBD:} #1}\xspace}
\newcommand{\todoR}[1]{\todo[fancyline,size=\small,color=orange!40]{\textbf{rc:} #1}\xspace}

\newcommand{\todoO}[1]{\todo[fancyline,size=\small,color=red!20]{\textbf{ok:} #1}\xspace}

\newcommand{\tododo}[1]{\todo[fancyline,size=\small,color=red!60]{\textbf{ToDO:} #1}\xspace}
\newcommand{\todoin}[1]{\todo[inline,size=\small,color=green!20]{\textbf{HERE:} #1}\xspace}


%\usepackage{pgf}https://preview.overleaf.com/public/pqghhhhwjqpw/images/986e3685291cb332a5fbe5a979b5045d2e649a43.jpeg
\usepackage{tikz}
%\usetikzlibrary{arrows,automata}
\usetikzlibrary{positioning,calc,backgrounds}

%%%%%%%MACROS%%%%%%%%
\newcommand{\Lmc}{\ensuremath{\mathcal{L}}\xspace}
\newcommand{\Imc}{\ensuremath{\mathcal{I}}\xspace}
\newcommand{\Jmc}{\ensuremath{\mathcal{J}}\xspace}
\newcommand{\Tmc}{\ensuremath{\mathcal{T}}\xspace}
\newcommand{\Amc}{\ensuremath{\mathcal{A}}\xspace}
\newcommand{\Rmc}{\ensuremath{\mathcal{R}}\xspace}
\newcommand{\Omc}{\ensuremath{\mathcal{O}}\xspace}
\newcommand{\Omcref}{\ensuremath{{\mathcal{O}^\textnormal{ref}}}\xspace}
\newcommand{\Omcfull}{\ensuremath{{\mathcal{O}^\textnormal{full}}}\xspace}
\newcommand{\Ontology}{\Omc}

\newcommand{\EL}{\ensuremath{\mathcal{E\!L}}\xspace}
\newcommand{\elpp}{\ensuremath{\mathcal{E\!L}^{++}}\xspace}

\newcommand{\Inf}{{\ensuremath{\mathsf{Inf}}}\xspace}
\newcommand{\qual}{{\ensuremath{\mathsf{IIC}}}\xspace}

\newcommand{\UpC}{{\ensuremath{\mathsf{UpCover}}}\xspace}
\newcommand{\DownC}{{\ensuremath{\mathsf{DownCover}}}\xspace}

\newcommand{\disjoint}{\ensuremath{\mathit{disjoint}}\xspace}
\newcommand{\self}{\ensuremath{\mathit{Self}}\xspace}
\newcommand{\less}[2]{\ensuremath{\leq #1~#2}\xspace}
\newcommand{\more}[2]{\ensuremath{\geq #1~#2}\xspace}
\newcommand{\nominal}[1]{\ensuremath{\{#1\}}\xspace}

\newcommand{\inv}{\ensuremath{\mathit{inv}}\xspace}
\newcommand{\refine}{\ensuremath{\mathop{\uparrow}}\xspace}
\newcommand{\corefine}{\ensuremath{\mathop{\downarrow}}\xspace}
%\DeclareMathOperator*{\argmax}{\mathsf{arg\,max}}
\DeclareMathOperator*{\argmax}{\mathsf{argmax}}
